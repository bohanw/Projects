\documentclass[twocolumn,fleqn,layout]{article}
\usepackage{alltt}

\pagestyle{plain}
\flushbottom
\sloppy
\begin{document}

\def\trule#1{\hbox{\vbox to 3mm{\vfill\hrule height0.4pt width#1\vskip2pt\hrule height0.4pt width#1\vfill}}}

\def\rrule#1{\hbox{\vbox to 3mm{\vfill\hrule height0.4pt width#1\vfill}}}


\def\Rule#1#2{\mbox{${\displaystyle\raise 3pt\hbox{$\;\;\;#1\;\;\;$}}
                          \over
                     {\displaystyle\lower5pt\hbox{$\;\;\;#2\;\;\;$}}$}}


% ---------------------------------------------------------------------
% Macros for little HOL sessions displayed in boxes.
%
% Usage: (1) \setcounter{sessioncount}{1} resets the session counter
%
%	 (2) \begin{session}\begin{verbatim}
%	      .
%	       < lines from hol session >
%	      .
%	     \end{verbatim}\end{session}
%
%            typesets the session in a numbered box.
% ---------------------------------------------------------------------

\newlength{\hsbw}
\setlength{\hsbw}{83mm}
\addtolength{\hsbw}{-\arrayrulewidth}
\addtolength{\hsbw}{-\tabcolsep}
\newcommand\HOLSpacing{12pt}

\newcounter{sessioncount}
\setcounter{sessioncount}{1}

\newenvironment{session}{\begin{flushleft}
 \begin{tabular}{@{}|c@{}|@{}}\hline
 \begin{minipage}[b]{\hsbw}
% \vspace*{-.5pt}
 \begin{flushright}
 \rule{0.01in}{.15in}\rule{0.3in}{0.01in}\hspace{-0.35in}
 \raisebox{0.04in}{\makebox[0.3in][c]{\footnotesize\sl \thesessioncount}}
 \end{flushright}
 \vspace*{-11.5mm}
 \begingroup}{\endgroup\end{minipage}\\ \hline
 \end{tabular}
 \end{flushleft}
 \stepcounter{sessioncount}}


\sloppy



\twocolumn[{\scriptsize \parbox{100mm}{Preprint of a paper
to appear in the Proceedings of the 1991 International Tutorial
and Workshop on the HOL Theorem Proving System, 27--30 August 1991,
Davis California (IEEE Computer Society Press).}}\vskip10mm
\begin{center}
{\Large\bf A Package for Inductive Relation Definitions in HOL}\\
\vskip24pt
T.\ F.\ Melham\\
\vskip12pt
University of Cambridge Computer Laboratory\\
New Museums Site, Pembroke Street\\
Cambridge, CB2 3QG, England.\\
\mbox{}\\
\vskip3mm
\end{center}]

\subsection*{\centering Abstract}

{\it\sloppy This paper describes a set of theorem proving tools based on a new
derived principle of definition in HOL, namely the introduction of relations
inductively defined by a set of rules.  Such inductive definitions abound in
computer science. Example application areas \mbox{include} reasoning about
structured operational semantics, type judgements, transition relations for
process algebras, \mbox{reduction} relations, and compositional proof systems.
The package described in this paper automates the derivation of certain
inductive definitions involved in these applications and provides the basic
tools needed for reasoning about the relations introduced by them.}

\vskip12pt
\section{Introduction}

The HOL user community has a strong tradition of taking a purely {\it
definitional\/} approach to using higher order logic. That is, the syntax of
the logic is extended with new notation not simply by postulating axioms to
give meaning to it, but rather by defining it in terms of existing expressions
of the logic that already have the required semantics.  The advantage of this
approach, as opposed to the axiomatic method, is that each of the primitive
rules of definition in the {\small HOL} logic---namely, constant definition,
constant specification, and type definition---is guaranteed to preserve
consistency.  The disadvantage is that these rules admit only
\mbox{definitions} that satisfy certain very restrictive rules of formation.
Definitions expressed in any other form must always be justified formally by
deriving them from equivalent, but possibly rather complex, primitive
definitions.

The {\small ML} metalanguage allows users to implement derived inference rules
in the {\small HOL} system and thus provides a facility for automating proofs
that justify derived rules of definition.  For example, recursive
\mbox{definitions} are not admitted by the primitive rules of definition of the
{\small HOL} logic.  But certain recursive type definitions and function
definitions are supported in the system by derived inference rules written in
{\small ML}~\cite{description,melham}.  The details of the primitive
definitions that underlie these rules are hidden from the user, and their
{\small ML} implementations are highly optimized. So these \mbox{derived}
principles of definition may simply be regarded as primitive by most users of
the system.

This paper describes a set of theorem-proving tools based on a new derived
principle of definition in {\small HOL} for defining relations inductively by a
set of rules. \mbox{Sections~\ref{ind-defs}} and~\ref{in-logic} give a general
introduction to the class of inductive definitions handled by the package and
explain the logical basis for these definitions.  The remaining sections
describe the {\small ML} functions provided by the package and briefly mention
some applications for which the package can be used.

\section{Inductive definitions}\label{ind-defs}

The following is a simple but typical example of a relation defined inductively
by a set of rules. (This example is taken from~\cite{pitts}.) Let $R \subseteq
A \times A$ be a binary relation on a set $A$.  The reflexive-transitive
closure of $R$ can be defined to be the least relation $R^{*} \subseteq A
\times A$ for which the following deduction rules hold.

\medskip

\[ \begin{array}[t]{@{}l}
   \hbox{{\small\bf R}\bf 1}\quad\Rule{}{R^{*}(x,y)}\;\; R(x,y) \\[6mm]
   \hbox{{\small\bf R}\bf 2}\quad\Rule{}{R^{*}(x,x)}\\[8mm]
   \hbox{{\small\bf R}\bf 3}\quad\Rule{R^{*}(x,z)\qquad R^{*}(z,y)}
         {R^{*}(x,y)}
\end{array} \]

\medskip

\noindent These rules state precisely the properties required of the
reflexive-transitive closure of the relation $R$.  Rule {{\small\bf R}\bf 1}
states that it must contain at least all the values in $R$; rule {{\small\bf
R}\bf 2} states that it must be reflexive; and rule {{\small\bf R}\bf 3} states
that it must be transitive.  The reflexive-transitive closure $R^{*}$ may
therefore simply be {\it defined\/} to be the least relation that satisfies
these conditions.  It then follows simply by definition that the rules
{{\small\bf R}\bf 1}, {{\small\bf R}\bf 2} and {{\small\bf R}\bf 3} are in fact
satisfied by $R^{*}$.  Moreover, it follows immediately that $R^{*}$ is a
subset of any other relation that satisfies these rules, since $R^{*}$ is
defined to be the {\it least\/} such relation.  This means that $R^{*}$
contains only those pairs of values that it must contain by virtue of
satisfying the rules. As will be discussed below, this property gives rise to
an induction principle for reasoning about the relation $R^{*}$.

The definition given above is valid because the rules {{\small\bf R}\bf 1},
{{\small\bf R}\bf 2}, and {{\small\bf R}\bf 3} make only positive statements
about the elements of $R^{*}$.  This guarantees that the least relation
satisfying these rules does exist.  In particular, if the rules have this form,
then one can show that the intersection of any set of relations that satisfy
the rules also satisfies the rules.  Moreover, at least one binary relation
satisfies the rules, namely the maximal relation $A \times A$. The `least' or
smallest relation that satisfies the rules may therefore legitimately be
defined to be the intersection of all such relations.

In general, an inductive definition of an $n$-place \mbox{relation} $R$
consists of a set of rules of the form:

\smallskip

\[ \Rule{R(t_1^1,\dots,t_n^1) \quad \cdots \quad R(t_1^i,\dots,t_n^i)}
        {R(t_1,\dots,t_n)}\;\;C_1\:\cdots\:C_j \]

\smallskip

\noindent The terms above the line are the {\it premisses\/} of the rule, each
of which makes a positive assertion of membership in the relation $R$.  The
term below the line, called the {\it conclusion\/} of the rule, likewise
asserts membership in $R$. The terms $C_1$,\dots,$C_j$ are {\it side
conditions\/} on the rule; these may be arbitrary propositions not involving
the relation $R$ being defined. A relation $R$ is {\it closed\/} under such a
rule if whenever the premisses and side conditions hold, the conclusion also
holds.  The relation {\it inductively defined\/} by a collection of such rules
is the least relation closed under all the rules.

\subsection{Rule induction}\label{rule-ind}

By virtue of its definition as the {\it least\/} relation closed under a set of
rules, every inductively defined relation comes with an associated induction
principle.  This principle of {\it rule induction\/} is essential for many
proofs involving such relations. (The term `rule induction' was coined by Glynn
Winskel in~\cite{winskel}).

The principle of rule induction for an inductively defined relation may be
stated briefly as follows. Let $R$ be an $n$-place relation inductively defined
by a set of rules, and suppose we wish to show that every element in $R$ has a
certain property $P$:

{\samepage
\begin{equation}\label{fact1}
\hbox{\rm
if}\;\;R(x_1,\dots,x_n)\;\;\hbox{then}\;\;P[x_1,\dots,x_n]
\end{equation}

\noindent Since $R$ is} \pagebreak[3] the least relation closed under the
rules, any relation $S$ which is also closed under the rules has the property
that $R \subseteq S$.  Now, let

\[S = \{(x_1,\dots,x_n) \mid P[x_1,\dots,x_n]\}\]

\noindent Then to prove the desired property of $R$, it suffices to show that
the relation $S$ is closed under the rules that define $R$. For if the relation
$S$ in fact is closed under the rules, then we have that $R \subseteq S$ and
therefore that every element of $R$ has the defining property of $S$---i.e.\
statement~(\ref{fact1}) holds of the relation $R$.

For the relation $R^{*}$ defined above, the principle of rule induction is
stated as follows. In order to prove that a property $P[x,y]$ holds for all
$x$ and $y$ for which $R^{*}(x,y)$, it suffices to show that:

\begin{itemize}

\item for all $x$ and $y$, $R(x,y)$ implies $P[x,y]$

\item for all $x$, $P[x,x]$

\item for all $x$, $y$, and $z$, $P[x,z]$ and $P[z,y]$ imply $P[x,y]$

\end{itemize}

\noindent This is an inductive form of argument: if the property $P$ holds in
the `base cases' corresponding to rules {{\small\bf R}\bf 1} and {{\small\bf
R}\bf 2}, and if $P$ is preserved by the rule {{\small\bf R}\bf 3} (the `step
case' of the induction), then every pair in $R^{*}$ has the property $P$.  A
similar induction principle holds for every relation inductively defined by a
set of rules.

\section{Inductive definitions in logic}\label{in-logic}


Inductive definitions are based on the concept of a relation being closed under
a set of rules.  Since rules are essentially implications---{\it if\/} the
premisses and side conditions hold, {\it then\/} the conclusion holds---it is
straightforward to express this concept in logic.

Consider, for example, the rules given above for reflexive-transitive closure.
Let $R : \alpha{\rightarrow}\alpha{\rightarrow}bool$ be a fixed but arbitrary
 relation on $\alpha$. (Here, a relation is represented by a curried function;
but we shall continue to speak loosely of a pair of values $x$ and $y$ as being
`in' the relation $R$ when $R\;x\;y$ holds.)  The following formula then
asserts that a relation $P : \alpha{\rightarrow}\alpha{\rightarrow}bool$ is
closed under the rules defining the reflexive-transitive closure of $R$:

\smallskip
\[\begin{array}[t]{@{}l}
 (\forall x\:y.\:R\;x\;y \supset P\;x\;y)\; \wedge \\
 (\forall x.\:P\;x\;x)\; \wedge \\
 (\forall x\:y.\:(\exists z.\:P\;x\;z \wedge P\;z\;y) \supset P\;x\;y)
\end{array}\]
\smallskip

\noindent Each rule is expressed by a quantified implication of its conclusion
by the conjunction of its premisses and side conditions.  A rule with no side
conditions or premisses is just represented by a universally quantified
assertion of its conclusion. Closure of a relation under any set of rules of
the form discussed above can be expressed in logic in a similar way.

Using this method of expressing the notion of \mbox{closure} under a set of
rules, one can define the {\it least\/} relation closed under a set of rules
simply by taking the intersection of all such relations.  For example, a
function

\[ {\sf Rtc} : (\alpha{\rightarrow}\alpha{\rightarrow}bool) \rightarrow
(\alpha{\rightarrow}\alpha{\rightarrow}bool) \]

\noindent that maps an arbitrary relation $R :
\alpha{\rightarrow}\alpha{\rightarrow}bool$ to its reflexive-transitive closure
${\sf Rtc}\;R$ can be defined in the {\small HOL} logic by the constant
definition:

\[\vdash \begin{array}[t]{@{}l}
\forall R\:x\:y.\:{\sf Rtc}\;R\;x\;y\;= \\
\quad\forall P.\:(\begin{array}[t]{@{}l}
  (\forall x\: y.\:R\;x\;y \supset P\;x\;y)\; \wedge \\
  (\forall x.\:P\;x\;x)\; \wedge \\
  (\forall x\: y.\:(\exists z.\:P\;x\;z \wedge P\;z\;y) \supset P\;x\;y)) \\
\quad\supset\\
P\;x\;y
\end{array}\end{array}\]

\noindent This definition states that a pair $x$ and $y$ is in the relation
${\sf Rtc}\;R$ exactly when it is in every relation $P$ closed under the rules
for reflexive-transitive closure. That is, ${\sf Rtc}\;R$ is \mbox{defined} to
be the intersection of all relations closed under these rules.  As will be
discussed in the section that follows, this indeed makes ${\sf Rtc}\;R$ the
least such relation, as required.


\subsection{Deriving the rules and rule induction}

Any relation intended to be defined inductively by a set of rules can be
defined formally in the {\small HOL} logic by a constant definition of the kind
illustrated by the {\sf Rtc} example given above.  Such a definition, however,
merely introduces the relation as the intersection of all relations that
satisfy the desired set of rules.  The proof obligations of a derived principle
of inductive definition are, first of all, to show that the resulting relation
in fact does satisfy these rules, and secondly to show that it is indeed the
least such relation.  It is these proof obligations which are automated by the
{\small HOL} inference rule described below in section~\ref{newind}.

In the case of the simple reflexive-transitive closure example, the first proof
obligation is to show that:

\[ \begin{array}[t]{@{}l}
   \vdash \forall R\:x\:y.\:R\;x\;y \supset {\sf Rtc}\;R\;x\;y \\[2mm]
   \vdash \forall R\:x.\:{\sf Rtc}\;R\;x\;x \\[2mm]
   \vdash \forall R\:x\:y.\:
        (\exists z.\:{\sf Rtc}\;R\;x\;z \wedge {\sf Rtc}\;R\;z\;y) \supset
        {\sf Rtc}\;R\;x\;y
\end{array}\]

\noindent That is, one must prove that the rules {{\small\bf R}\bf 1},
{{\small\bf R}\bf 2}, and {{\small\bf R}\bf 3} follow from the somewhat
indirect formal definition of the relation ${\sf Rtc}\;R$ given in the previous
section.  The second proof obligation is to show that ${\sf Rtc}\;R$ is the
least relation that satisfies these rules:

\[\vdash \begin{array}[t]{@{}l}
\forall R\:P.\:\begin{array}[t]{@{}l}
  (\begin{array}[t]{@{}l}
  (\forall x\:y.\:R\;x\;y \supset P\;x\;y)\; \wedge \\
  (\forall x.\:P\;x\;x)\; \wedge \\
  (\forall x\:y.\:(\exists z.\:P\;x\;z \wedge P\;z\;y)
    \supset P\;x\;y))\end{array}\\
\quad\supset\\
\forall x\:y.\: {\sf Rtc}\;R\;x\;y \supset P\;x\;y
\end{array}\end{array}\]

\noindent This is the principle of rule induction for ${\sf Rtc}\;R$.  These
four theorems constitute a complete statement of the defining properties of
reflexive-transitive closure. All four can be proved fully automatically in
{\small HOL} by the derived inference rule described in the next section.

\section{Automation}\label{newind}

The main component of the inductive definitions package is an {\small ML}
function that takes as an argument a list of rules and automatically proves the
defining properties of the \mbox{relation} inductively defined by them.  More
precisely, this derived {\small HOL} inference rule builds a term that denotes
the least relation closed under the rules using the intersection construction
described in the previous section.  A constant is then introduced (via a
constant specification) to name this relation.  The result is a set of theorems
stating that the newly-defined relation is the least relation closed under the
rules supplied by the user.

The {\small ML} function that implements this principle of inductive definition
is:

\medskip

\noindent\begin{tabular}{@{\hskip\mathindent}l@{\hskip4.6mm}l@{}}
\verb!new_inductive_definition! & \mbox{} \\
\verb! : bool ->! & ({\it infix flag\/})\\
\verb!   string ->! & ({\it defn.\ name\/})\\
\verb!   (term # term list) ->! & ({\it pattern\/})\\
\verb!   (term list # term) list ->! & ({\it rules\/})\\
\verb!   (thm list # thm)! & ({\it result\/})
\end{tabular}

\medskip

\noindent The first argument to this function is a boolean flag which indicates
if the constant that is defined is to have infix syntactic status.  The second
argument is the name under which the resulting definition will be saved on
disk.  The third argument is a `pattern' that supplies information which is
needed because this {\small ML} function can be used to define classes of
inductively defined relations, rather than just single instances of these
relations. Details of the purpose and format of this pattern will be explained
later.  The final argument is a list of rules, each of which is represented by
a pair of the form:

\[ \hbox{\verb!([!}\,
\hbox{\it premisses and side conditions\/}\,\hbox{\verb!], !}
 \hbox{\it conclusion\/}\hbox{\verb!)!} \]

\noindent The first component is a list of the premisses and side conditions,
which may be arranged in any order.  The second component is the conclusion of
the rule.  Side conditions can be arbitrary boolean terms, provided they do not
mention the relation being defined.  The premisses and conclusion must be
positive assertions of membership in the relation being defined.  The precise
form that these assertions must take is explained later, but roughly speaking
the premisses and conclusion of a rule must be terms of form
\verb!"!$R\;\,t_1\;\,\dots\;\,t_n$\verb!"!, where

\[ R\; \hbox{\verb!:!}\; \sigma_1 \;\hbox{\verb!->!}\; \dots
\;\hbox{\verb!->!}\; \sigma_n \;\hbox{\verb!->!}\; \hbox{\verb!bool!}\]

\noindent is a variable representing the $n$-place relation that is to be
defined, and each $t_i \hbox{\verb!:!} \sigma_i$ is an arbitrary term not
containing $R$.

Given an infix flag, a name, a pattern, and a list of rules, the {\small ML}
function \verb!new_inductive_definition! automatically proves the existence of
the least relation that satisfies these rules.  A constant is then introduced
to denote this relation using a constant specification, the result of which is
saved on disk under the supplied name. The value returned is a pair consisting
of a list of theorems which state that the newly-defined relation satisfies the
rules, together with a \mbox{theorem} asserting rule induction for the
relation.  These theorems give a complete statement of the defining properties
for the least relation closed under the specified set of rules.

\subsection{A simple example}

The following example {\small HOL} session shows how the function
\verb!new_inductive_definition! can be used to inductively define the set of
even natural numbers.

\begin{session}\begin{verbatim}
#let (rules,ind) =
   let Even = "Even:num->bool" in
   new_inductive_definition false `Even`
   ("^Even n", [])

   [ [
     % ----------------------------- % ],
                 "^Even 0"           ;

     [           "^Even n"
     % ----------------------------- % ],
               "^Even (n+2)"        ];;
\end{verbatim}\end{session}


\noindent The first rule in this definition states that \verb!0! is an even
natural number, and the second rule states that if \verb!n! is even then
\verb!n+2! is also even.  (Antiquotation and {\small ML} comments are used to
give a readable presentation of these rules.)  Since the even natural numbers
are \mbox{exactly} those numbers obtainable from zero by adding multiples of
two, these rules inductively define `\verb!Even n!' such that it holds
\mbox{precisely} when \verb!n! is even.

The value supplied for the pattern in this example is the pair
\verb!("Even n",[])!.  The first component of this pair indicates that
the constant to be defined, namely \verb!Even!, is a one-place
function with typical argument \verb!n!.  In general, the second
component of a pattern is a non-empty list only when a {\it class\/}
of relations is being defined (see below).  In this example,
{\verb!Even!} is a single inductively-defined predicate, and the list
component of the pattern is therefore empty.


When the definition shown in box 1 is evaluated,
\verb!new_inductive_definition! automatically proves the existence of the least
predicate closed under the given list of rules and then defines the constant
\verb!Even! to denote this predicate.  The following automatically-proved
theorems about \verb!Even! are then returned:

\begin{session}\begin{alltt}
rules =
[\(\vdash\) Even 0;
 \(\vdash\) \(\forall\,\)n. Even n \(\supset\) Even(n + 2)] : thm list
ind =
\(\vdash\) \(\forall\,\)P. P 0 \(\wedge\) (\(\forall\,\)n. P n \(\supset\) P(n + 2)) \(\supset\)
      \(\,\)(\(\forall\,\)n. Even n \(\supset\) P n)
\end{alltt}\end{session}


\noindent The theorems bound to the {\small ML} identifier \verb!rules! state
that the required rules hold of the predicate \verb!Even!. And the rule
induction theorem bound to \verb!ind! states that the set of numbers for which
\verb!Even! holds is the least set that satisfies these rules.

An analogous set of defining theorems can be proved automatically for any
particular relation inductively defined by a list of rules.  The next section
shows how this derived principle of inductive definition in {\small HOL} can
also be used to define a parameterized class of relations.

\subsection{Defining a class of relations}

The constant {\sf Rtc} defined in section~\ref{in-logic} is not itself an
inductively-defined relation, but rather a function that maps an arbitrary
relation $R$ to an inductively-defined relation ${\sf Rtc}\;R$.  The function
{\sf Rtc} therefore represents an entire class of inductively-defined
relations, one for each possible value of $R$.

The information that is required by the derived rule
\verb!new_inductive_definition! in order to handle the definition of such
functions is supplied by its pattern argument.  In the general case, a pattern
is a pair of the following form:

\[\hbox{\verb!("!}R\;v_1\;\dots\;v_n\hbox{\verb!",!}\;
 \hbox{\verb!["!}v_i\hbox{\verb!";!}\dots\hbox{\verb!;"!}v_j\hbox{\verb!"])!}
\]

\noindent The first component of the pattern is an application of the $n$-place
curried function that is to be defined (in this case, $R$) to $n$ distinct
variables $v_1$, \dots, $v_n$.  The second component is a list of those
variables that occur at the positions in this application which correspond to
the parameters of class of inductively-defined relations, rather than to the
actual arguments to these relations.

An example of the role of the pattern argument in defining a class of relations
is provided by the following definition of reflexive-transitive closure in
{\small HOL}.

\begin{session}\begin{verbatim}
#let (rules,ind) =
   let Rtc = "Rtc:(*->*->bool)->*->*->bool"
 in
   new_inductive_definition false `Rtc`
   ("^Rtc R x y", ["R:*->*->bool"])

   [ [      "R (x:*) (y:*):bool"
     % ----------------------------- % ],
               "^Rtc R x y"          ;

     [
     %------------------------------ % ],
               "^Rtc R x x"          ;

     [  "^Rtc R x z";  "^Rtc R z y"
     %------------------------------ % ],
               "^Rtc R x y"          ];;
\end{verbatim}\end{session}

\noindent The pattern in this case is the pair:

\medskip

\noindent\hskip\mathindent\verb!("Rtc R x y", ["R:*->*->bool"])!

\medskip

\noindent The first component of this pattern specifies that the function
\verb!Rtc! is to take three arguments in total---a \mbox{relation} \verb!R!,
and two values \verb!x! and \verb!y!.  The \mbox{second} part of the pattern
(the list containing just \verb!R!) specifies that the relation argument
\verb!R! is to be a parameter to the class of inductively-defined relations
that will be represented by \verb!Rtc!.  The remaining variables \verb!x! and
\verb!y! are assumed to indicate the positions of actual arguments to the
predicate that represents these relations.

The result of evaluating this inductive definition in {\small HOL} is the
following collection of theorems:

\begin{session}\begin{alltt}
rules =
[\(\vdash\) \(\forall\,\)R x y. R x y \(\supset\) Rtc R x y;
 \(\vdash\) \(\forall\,\)R x. Rtc R x x;
 \(\vdash\) \(\forall\,\)R x y. (\(\exists\,\)z. Rtc R x z \(\wedge\) Rtc R z y)
              \(\supset\)
            Rtc R x y] : thm list
ind =
\(\vdash\) \(\forall\,\)R P.
   (\(\forall\,\)x y. R x y \(\supset\) P x y) \(\wedge\)
   (\(\forall\,\)x. P x x) \(\wedge\)
   (\(\forall\,\)x y. (\(\exists\,\)z. P x z \(\wedge\) P z y) \(\supset\) P x y)
      \(\supset\)
   (\(\forall\,\)x y. Rtc R x y \(\supset\) P x y)
\end{alltt}\end{session}

\noindent Here, the {\small ML} variable \verb!rules! has been bound to a list
of theorems which state the three rules that inductively define the
reflexive-transitive closure of a relation.  The \mbox{theorem} \verb!ind!
states the corresponding principle of rule induction for an inductively-defined
relation \verb!Rtc R!.

\subsection{Stating premisses and conclusions}

In addition to the use of the pattern argument, the \verb!Rtc! example also
illustrates a restriction on the form in which the premisses and conclusions of
rules must be supplied to \verb!new_inductive_definition!.  As was mentioned
above, premisses and conclusions must be positive assertions of membership of
the form

\medskip

\noindent\hskip\mathindent\verb!"!$R\;\,t_1\;\,\dots\;\,t_n$\verb!"!

\medskip

\noindent where $R$ is a variable that stands for the function to be defined.
The restriction is that some of the terms among the arguments $t_1$, \dots,
$t_n$ in such an \mbox{assertion} must be variables---namely, the terms that
occur at \mbox{positions} which, according to the supplied pattern,
\mbox{correspond} to the parameters of a class of relations.  In particular,
the terms that occur at these positions must be the same variables given in the
pattern itself.

The rules for reflexive-transitive closure shown in box 3 conform to
this restriction.  Here, the pattern indicates that in the typical
assertion of membership \verb!"Rtc R x y"! (i.e.\ the first component
of the pattern), the variable \verb!R! marks the position of a
parameter to the class of relations to be defined.  Every premiss and
conclusion mentioned in the rules must therefore be a term of the form
$\hbox{\verb!"Rtc R!}\;\,t_1\;\,t_2\hbox{\verb!"!}$, where the
arguments $t_1$ and $t_2$ may be arbitrary terms but the parameter
\verb!R! must be the variable given in the pattern.

\section{A tactic for rule induction}

The inductive definitions package in {\small HOL} includes a number of
auxiliary functions that support reasoning about inductively-defined relations,
in addition to the derived rule of definition itself.  The most important of
these is the following general tactic for goal-directed proofs by rule
induction:

\medskip

\noindent\begin{tabular}{@{\hskip\mathindent}l@{\hskip12.7mm}l@{}}
\verb!RULE_INDUCT_THEN! & \mbox{} \\
\verb! : thm ->! & ({\it induction thm\/})\\
\verb!   (thm -> tactic) ->! & ({\it premiss cont.\/})\\
\verb!   (thm -> tactic) ->! & ({\it side cond.\ cont.\/})\\
\verb!   tactic! & ({\it result\/})
\end{tabular}

\medskip

\noindent The first argument to this function is the rule \mbox{induction}
theorem returned by \verb!new_inductive_definition! for a given
inductively-defined relation.  Like the general structural induction tactic in
{\small HOL}, the rule induction tactic is parameterized by functions that
determine what is done with induction hypotheses. These may be either premisses
or side conditions, and the user may wish to treat these two kinds of induction
hypotheses differently. Two separate theorem continuations are therefore
supplied as the second and third arguments to the function
\verb!RULE_INDUCT_THEN!.

Given the rule induction theorem for an inductively-defined $n$-ary relation
$R$, the function described above returns a specialized rule induction tactic
that reduces goals of the form:

\[ \hbox{\verb!"!}\forall x_1\;\dots\;x_n\hbox{\verb!.!}\;
  R \;x_1\;\dots\;x_n \supset P[x_1,\dots,x_n]\hbox{\verb!"!} \]

\noindent to the subgoal(s) of proving that the property $P$ is preserved by
the rules that inductively define $R$.  The rule induction theorem for
\verb!Rtc!, for example, is:

\begin{session}\begin{alltt}
#ind;;
\(\vdash\) \(\forall\,\)R P.
   (\(\forall\,\)x y. R x y \(\supset\) P x y) \(\wedge\)
   (\(\forall\,\)x. P x x) \(\wedge\)
   (\(\forall\,\)x y. (\(\exists\,\)z. P x z \(\wedge\) P z y) \(\supset\) P x y)
      \(\supset\)
   (\(\forall\,\)x y. Rtc R x y \(\supset\) P x y)
\end{alltt}\end{session}

\noindent A rule induction tactic for \verb!Rtc! can be constructed from this
theorem by making the simple {\small ML} definition:

\begin{session}\begin{alltt}
#let Rtc_INDUCT_TAC =
   RULE_INDUCT_THEN ind
      ASSUME_TAC ASSUME_TAC;;
Rtc_INDUCT_TAC = - : tactic
\end{alltt}\end{session}

\noindent The use of \verb!ASSUME_TAC! in this definition means that the
induction hypotheses arising from the premisses and side conditions of the
rules are to be added to the assumptions of the subgoals that are generated.
The resulting rule induction tactic for \verb!Rtc!  is described by:

\bigskip

\noindent\begin{tabular}{@{\hskip\mathindent}c@{}}
$\Gamma\;\:$\verb!?-!$\;\:\forall x\:y$\verb!. Rtc!$\;R\;x\;y \supset P[x,y]$\\
\trule{62mm}\\
$\Gamma \cup \{R\;x\;y\}\;\:$\verb!?-!$\;\:P[x,y]$\\[1mm]
$\Gamma\;\:$\verb!?-!$\;\:\forall x$\verb!.!$\;P[x,x]$\\[1mm]
$\Gamma\;{\cup}\;\{P[x,z],\:P[z,y]\}\;\:$\verb!?-!$\;\:P[x,y]$
\end{tabular}

\bigskip

\noindent This tactic implements the induction scheme described above in
section~\ref{rule-ind}. It reduces the goal of proving that a property $P[x,y]$
holds for all pairs $x$ and $y$ related by $\hbox{\verb!Rtc!}\;R$ to showing
that this property is preserved by the rules that inductively define this
relation.

\subsection{An example}

The following session shows how the rule induction tactic for \verb!Rtc!
constructed in the previous section can be used to prove a simple theorem about
this relation.  The aim is to show that the reflexive-transitive closure of a
symmetric relation is also symmetric.  The proof begins by using the {\small
HOL} subgoal package (see~\cite{description}) to set up an appropriate goal to
be proved:

\smallskip

\begin{session}\begin{alltt}
#set_goal
   (["\(\forall\,\)x:*. \(\forall\,\)y. R x y \(\supset\) R y x"],
     "\(\forall\,\)x:*. \(\forall\,\)y. Rtc R x y \(\supset\) Rtc R y x");;
"\(\forall\,\)x y. Rtc R x y \(\supset\) Rtc R y x"
    [ "\(\forall\,\)x y. R x y \(\supset\) R y x" ]

() : void
\end{alltt}\end{session}

\smallskip

\noindent The assumption of the goal is that the relation \verb!R! is
symmetric, and the conclusion states that the closure \verb!Rtc R! is also
symmetric.  The conclusion of the goal is in precisely the right form for a
proof by rule induction using the induction tactic described above.  Applying
this tactic results in:

\smallskip

\begin{session}\begin{alltt}
#expand Rtc_INDUCT_TAC;;
OK..
3 subgoals
"Rtc R y x"                       {\rm({\it{subgoal 1\/}})}
    [ "\(\forall\,\)x y. R x y \(\supset\) R y x" ]
    [ "Rtc R z x" ]
    [ "Rtc R y z" ]

"\(\forall\,\)x. Rtc R x x"                   \(\!\){\rm({\it{subgoal 2\/}})}
    [ "\(\forall\,\)x y. R x y \(\supset\) R y x" ]

"Rtc R y x"                       {\rm({\it{subgoal 3\/}})}
    [ "\(\forall\,\)x y. R x y \(\supset\) R y x" ]
    [ "R x y" ]

() : void
\end{alltt}\end{session}

\smallskip

\noindent Subgoals 1 and 2 are trivial, since the relation \verb!Rtc R! is
transitive and reflexive by definition. The tactic proofs for these subgoals
can simply use the rules shown above in box~4. The proof of subgoal 3 is also
easy. The proposition \verb!"R y x"! follows immediately from the two
assumptions of the subgoal; and this proposition \mbox{together} with the fact
that

\bigskip

\noindent\hskip\mathindent$\vdash\forall\,$\verb!R x y. R x y !$\supset
$\verb! Rtc R x y!

\bigskip

\noindent directly entail the required conclusion.

The proof sketched above is a trivial example of the kind of reasoning
sometimes referred to as induction over the structure (or the depth) of
derivations in a deductive system stated by a set of rules.  This form of
inductive \mbox{argument}, which is very common in certain \mbox{areas} of
theory (for example, operational semantics and process algebras), is made
directly accessible in {\small HOL} by the tactic described in this section.

\section{Tactics and inference rules}

In addition to the rule induction tactic described above, the inductive
definitions package also provides mechanized support for generating tactics
from the theorems that state the rules for an inductively-defined relation.
This takes the form of an {\small ML} function:

\medskip

\noindent\hskip\mathindent\verb!RULE_TAC : thm -> tactic!

\medskip

\noindent The theorem argument to this function is expected to be a rule
expressed in the form proved by the derived principle of inductive definition
described in section~\ref{newind}. Given such a theorem, \verb!RULE_TAC!
constructs a tactic that inverts the rule stated by it.  The resulting tactic
reduces goals that match the conclusion of the rule to subgoals consisting of
the corresponding instances of its premisses and side conditions.

Consider, for example, the theorem which states the transitivity rule for
\verb!Rtc!:

\[\vdash \forall\,\hbox{\verb!R x y. !}\begin{array}[t]{@{}l}%
\hbox{\verb!(!}\exists\,
\hbox{\verb!z. Rtc R x z !}\wedge\hbox{\verb! Rtc R z y)!}\\
\quad \supset \\
\hbox{\verb!Rtc R x y!}
\end{array} \]

\noindent When applied to this theorem, the function \verb!RULE_TAC! returns
the tactic described by:

\bigskip

\noindent\begin{tabular}{@{\hskip\mathindent}c@{}}
$\Gamma\;\:$\verb!?-!$\;\:$\verb!Rtc!$\;R\;x\;y$ \\
\trule{62mm}\\
$\Gamma\;\:$\verb!?-!$\;\:\exists z
$\verb!.!$\;$\verb!Rtc!$\;R\;x\;z\;\:\wedge\;\;$\verb!Rtc!$\;R\;z\;y$
\end{tabular}

\bigskip

\noindent This tactic can then be used in goal-directed proofs about membership
in the inductively-defined relation {\verb!Rtc!$\;R$}.  The other two rules
that define \verb!Rtc!$\;R$ can also be converted into tactics using the
function \verb!RULE_TAC!.  The result is a complete set of {\small HOL} tactics
for goal-directed proofs in the deductive system comprising the three rules
that define reflexive-transitive closure.

It is intended that the inductive definitions package will also include a
function that maps rules stated as theorems to forward inference rules in
{\small HOL} (i.e.\ to {\small ML} functions).  For example, the transitivity
theorem shown above can be used to implement the following derived
inference rule:

\bigskip

\noindent\begin{tabular}{@{\hskip\mathindent}c@{}}
$\Gamma_1\;{\vdash}\;$\verb!Rtc!$\;R\;x\;z$\qquad
$\Gamma_2\;{\vdash}\;$\verb!Rtc!$\;R\;z\;y$\\
\rrule{62mm}\\
$\Gamma_1 \cup \Gamma_2\;{\vdash}\;$\verb!Rtc!$\;R\;x\;y$
\end{tabular}

\bigskip

\noindent Any rule expressed as a theorem of the form proved by the derived
principle of inductive definitions can likewise be converted into a forward
inference rule.  A function that \mbox{automatically} constructs such rules has
not yet been implemented, partly because it has not been found necessary for
the applications done so far (see section~\ref{appl}).  For completeness,
however, the author intends in future to add this function to the inductive
definitions package.

\section{Case analysis}

The final major component of the {\small HOL} package for inductive definitions
is an {\small ML} function that proves an \mbox{exhaustive} case analysis
theorem for any given relation inductively defined by a set of rules.  The name
and type of this function are:

\medskip

\noindent\hskip\mathindent\verb!derive_cases_thm : (thm list # thm) -> thm!

\medskip

\noindent The arguments to this function are the list of rules satisfied by an
inductively defined relation, together with its rule induction theorem. (These
are precisely the defining theorems which are proved and returned by
\verb!new_inductive_definition!.)  When supplied with these theorems,
\verb!derive_cases_thm! proves that if an assertion of membership in the
relation holds, then it holds only by virtue of the fact that one of the rules
can be used to derive it.  This allows one to drive the rules that define a
relation `backwards', inferring from the conclusion of one of the rules that
the premisses and side conditions hold.

The following interaction with the {\small HOL} system shows the theorem proved
by \verb!derive_cases_thm! for the \verb!Rtc! example introduced above.  The
{\small ML} variables \verb!rules! and \verb!ind! are assumed to have the
bindings shown above in box~4.

\smallskip

\begin{session}\begin{alltt}
#derive_cases_thm (rules,ind);;
\(\vdash\) \(\forall\,\)R x y.
    Rtc R x y \(\supset\)
      R x y   \(\vee\)
      (y = x) \(\vee\)
      (\(\exists\,\)z. Rtc R x z \(\wedge\) Rtc R z y)
\end{alltt}\end{session}

\smallskip

\noindent Roughly speaking, the resulting theorem states that if
\verb!Rtc R x y! holds, then either:

\begin{itemize}

\item it is derivable by the inclusion rule {{\small\bf R}\bf 1}, in which
case \verb!x! and \verb!y! are related by \verb!R!; or

\item it is derivable by the reflexivity rule {{\small\bf R}\bf 2}, in which
case \verb!x! and \verb!y! are equal; or

\item it is derivable by the transitivity rule {{\small\bf R}\bf 3}, in which
case there must be an intermediate value \verb!z! such that \verb!Rtc R x z!
and \verb!Rtc R z y!.

\end{itemize}

\noindent A similar theorem can be proved automatically for any relation
defined inductively using the package. Work is currently underway to strengthen
this theorem from an implication to an equation, so that it can be used for
rewriting.

\section{Applications}\label{appl}

In a joint project with Juanito Camilleri, a set of example proofs has
been developed to illustrate the potential for applications of the inductive
\mbox{definitions} package.  These examples include: the definition of an
operational semantics for a simple programming language and a proof that its
evaluation relation is \mbox{deterministic}; the definition of a reduction
relation for combinatory logic and a proof that it has the Church-Rosser
property; a definition of \mbox{provability} in a Hilbert style proof system
for minimal intuitionistic logic; the definition of a type system for
combinatory logic and a proof of the Curry-Howard isomorphism for typed
combinatory logic and minimal intuitionistic logic; and definitions of the
trace and transition semantics for a simple process algebra, \mbox{together}
with the proof of a formal statement of the relationship between them. A report
on this work is in preparation, and the {\small HOL} source code for the
examples will be made available to interested users.

\newpage

\begin{thebibliography}{9}

\bibitem{description}
DSTO, The University of Cambridge, and SRI
\mbox{International}, {\it The HOL System: DESCRIPTION} (1991).

\bibitem{melham}
T.\ F.\ Melham, `Automating Recursive Type Definitions
in Higher Order Logic',
in: {\it Current Trends in Hardware Verification and
Automated Theorem Proving\/}, edited by G.\ Birtwistle
and P.A.\ Subrahmanyam
(Springer-Verlag, 1989), pp.\ 341--386.

\bibitem{pitts}
A.\ M.\ Pitts, `Semantics of Programming Languages',
unpublished lecture notes, University of Cambridge Computer Laboratory
(October 1989).

\bibitem{winskel}
G.\ Winskel, `Introduction to the Formal Semantics of
Programming Languages', unpublished lecture notes, University of Cambridge
Computer Laboratory (October 1985).

\end{thebibliography}

\end{document}
